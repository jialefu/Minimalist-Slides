\documentclass[8pt]{ctexbeamer}

% 只需要在这里修改颜色,其他地方会自动引用
\definecolor{themegreen}{RGB}{48,95,33}
\definecolor{themered}{RGB}{95,33,48}

\colorlet{darkgray}{black!70}
\colorlet{stronghighlightcolor}{themered!150!black}
\colorlet{themecolor}{themegreen}
\usepackage{xcolor}
\usepackage{graphicx}
\usepackage[export]{adjustbox}
\usepackage{stackengine}
\usepackage{xcolor}
\usepackage{pifont}
\usepackage[framemethod=TikZ]{mdframed}
\usepackage{amsmath}
\usepackage{array}
\usepackage{listings}
\usepackage{tcolorbox}
\usepackage[backend=bibtex, style=numeric]{biblatex}
\usepackage{bm}
\usepackage{tabularx}
\usepackage{booktabs}
\usepackage{multirow}
\usepackage{ulem}
% Beamer 模板设置
\setbeamertemplate{navigation symbols}{}
\setbeamertemplate{footline}[page number]
\setbeamertemplate{section in toc}[sections numbered]

% Itemize 格式设置
\setbeamertemplate{itemize item}[circle] % 项目符号设置为圆形
\setbeamertemplate{itemize subitem}{$\circ$} % 子项目符号设置为圆形
\setbeamertemplate{itemize subsubitem}[circle] % 子子项目符号设置为圆形

% Enumerate 格式设置
\setbeamertemplate{enumerate subitem}{\alph{enumii}.} % 子项目符号设置为小写字母
\setbeamertemplate{enumerate subsubitem}{\roman{enumiii}.} % 子子项目符号设置为罗马数字

% 颜色设置
\setbeamercolor{frametitle}{bg=, fg=themecolor}
\setbeamercolor{block title}{bg=themecolor!8, fg=themecolor}
\setbeamercolor{description item}{fg=themecolor}
\setbeamercolor{section in toc}{fg=themecolor, bg=}
\setbeamercolor{subsection in toc}{fg=black!80}
\setbeamercolor{caption name}{fg=themecolor}
\setbeamercolor{bibliography entry author}{fg=black}
\setbeamercolor{bibliography entry title}{fg=black}
\setbeamercolor{bibliography entry journal}{fg=black}
\setbeamercolor{bibliography entry note}{fg=black}
\setbeamercolor{bibliography item}{fg=themecolor}
\setbeamercolor{itemize item}{fg=themecolor} % 项目符号颜色设置
\setbeamercolor{itemize subitem}{fg=themecolor} % 子项目符号颜色设置
\setbeamercolor{itemize subsubitem}{fg=themecolor} % 子子项目符号颜色设置
\setbeamercolor{enumerate item}{fg=themecolor} % 枚举项颜色设置
\setbeamercolor{enumerate subitem}{fg=themecolor} % 枚举子项颜色设置
\setbeamercolor{enumerate subsubitem}{fg=themecolor} % 枚举子子项颜色设置

% 字体设置
\setbeamerfont{frametitle}{size=\large, family=\kaishu}
\setbeamerfont{framesubtitle}{size=\normalsize, family=\kaishu}
\setbeamerfont{block title}{size=\large, family=\kaishu}
\setbeamerfont{section in toc}{size=\large, family=\kaishu}
\setbeamerfont{subsection in toc}{size=\normalsize}
\setbeamerfont{caption name}{family=\kaishu, size=\normalsize}
\setbeamerfont{caption}{family=\kaishu, size=\normalsize}
\setbeamerfont{footnote}{family=\kaishu, size=\tiny}
\setbeamerfont{description item}{size=\normalsize, family=\kaishu}

% 页脚模板设置
\setbeamertemplate{footline}{%
  \begin{beamercolorbox}[wd=\paperwidth, ht=2.25ex, dp=1ex, leftskip=0.5cm, rightskip=0.5cm]{title in head/foot}%
    \hfill{\textcolor{black!70}{\insertframenumber/\inserttotalframenumber}}\hspace*{1mm}\vspace*{2mm}
  \end{beamercolorbox}%
}

% 自定义标题模板
\defbeamertemplate*{frametitle}{}[1][]{
  \nointerlineskip%
  \vspace*{3mm}
  \hspace*{-1mm}
  \begin{beamercolorbox}[sep=0.3cm, wd=\paperwidth, leftskip=0.5cm, rightskip=0.5cm]{frametitle}
    {\usebeamerfont{frametitle}\insertframetitle}
    {\usebeamerfont{framesubtitle}\color{black!80}\insertframesubtitle
    \hfill
    \raisebox{-1mm}{\includegraphics[width=13mm]{figs/logo.jpg}}
    }\par
    \vskip-1.5ex
    \begin{tikzpicture}[remember picture, overlay]
      \draw[line width=0.2pt] (0,0) -- (\textwidth+4mm,0);
    \end{tikzpicture}
  \end{beamercolorbox}
  \vspace*{-7mm}
}

% 设置段落间距
\setlength{\parskip}{0.5em}

% 参考文献
\bibliography{reference.bib}
\normalem

% 数学字体主题设置
\usefonttheme[onlymath]{serif}

% 表格环境设置
\AtBeginEnvironment{table}{\setlength{\parskip}{0em}}

\newmdenv[
    hidealllines=true,
    leftline=true,
    linewidth=1pt,
    linecolor=darkgray,
    fontcolor=darkgray,
    innertopmargin=1pt,
    innerbottommargin=1pt,
    font=\kaishu\normalsize,
]{myquote}

\lstset{
    basicstyle=\ttfamily\color{stronghighlightcolor},
}

\lstdefinestyle{python}{
    language=Python,
    basicstyle=\normalsize\ttfamily,
    keywordstyle=\color{themecolor},
    stringstyle=\color{themered!150!black},
    commentstyle=\color{black!50},
    showstringspaces=false,
    backgroundcolor=\color{white},
    frame=single,
    framerule=0.25pt,
    framesep=3pt,
    rulecolor=\color{themecolor},
    numbers=left,
    numberstyle=\tiny\color{darkgray},
    numbersep=5pt,
    breaklines=true,
    escapeinside={(*@}{@*)}
}

\lstdefinestyle{latex}{
    language=[LaTeX]TeX,
    basicstyle=\normalsize\ttfamily,
    keywordstyle=\color{themecolor},
    stringstyle=\color{themered!150!black},
    commentstyle=\color{black!50},
    showstringspaces=false,
    backgroundcolor=\color{white},
    frame=single,
    framerule=0.25pt,
    framesep=3pt,
    rulecolor=\color{themecolor},
    numbers=left,
    numberstyle=\tiny\color{darkgray},
    numbersep=5pt,
    breaklines=true,
    escapeinside={(*@}{@*)}
}

\newcommand{\makesection}[1][0.4]{
    \begin{frame}[plain]
        \centering
        \vspace*{7mm}
        \begin{tabular}{l}
            {\color{themecolor}
            \huge
            \insertsectionnumber.\ \insertsection}\\
            \begin{tikzpicture}
                \fill[black!30] (0,0) rectangle (#1\textwidth, 1pt);
                \fill[themecolor] (0,0) rectangle (#1\textwidth/2, 1pt);
            \end{tikzpicture}
        \end{tabular}
    \end{frame}
}

\newcommand{\titlecolorbox}[1]{
    \begin{tcolorbox}[colback=themecolor!8, colframe=themecolor, arc=1mm, boxrule=0.2pt, width=0.97\textwidth]
        \color{themecolor} \LARGE \centering
        #1
    \end{tcolorbox}
}

\newcommand{\highlight}[1]{\textcolor{themecolor}{#1}}
\newcommand{\ulhighlight}[1]{\textcolor{themecolor}{\color{themecolor}\uline{#1}}}
\newcommand{\stronghighlight}[1]{\textcolor{stronghighlightcolor}{#1}}
\newcommand{\ulstronghighlight}[1]{\textcolor{stronghighlightcolor}{\color{stronghighlightcolor}\uline{#1}}}


\begin{document}

\kaishu

% 这里设置了两种封面,可以根据需要选择其中一种

\begin{frame}[plain]

    \centering
    \vspace{13mm}
    \titlecolorbox{简约风格汇报答辩模板}

    \vspace{7mm}
    \begin{tabular}{>{\color{themecolor}}r@{\hspace{3pt}}l}
        学生 & 张三 \\
        指导老师 & 李四 \\
    \end{tabular}

    \vspace{4mm} 
    \today

\end{frame}

\begin{frame}[plain]

    \setlength{\parskip}{0.2em}
    \vspace{13mm}
    {\color{themecolor}\huge
    简约风格汇报答辩模板
    }

    \begin{tikzpicture}
        \fill[black!50] (0,0) rectangle (0.98\textwidth, 0.5pt);
        \fill[themecolor] (0,0) rectangle (0.98\textwidth/2, 0.5pt);
    \end{tikzpicture}

    \vspace{7mm}
    \begin{tabular}{>{\color{themecolor}}r@{\hspace{3pt}}l}
        学生 & 张三 \\
        指导老师 & 李四 \\
    \end{tabular}

    \vspace{4mm} 
    \today

\end{frame}


\begin{frame}
    \frametitle{目录}
    \setlength{\parskip}{0.2em}
    \tableofcontents
\end{frame}

\section{模板介绍}

\makesection

\subsection{模板介绍及文件结构}

\begin{frame}{\insertsection}{\insertsubsection}

    本模板基于\lstinline|ctexbeamer|,采用了清新简约的设计风格,适合用于学术汇报、答辩等场合。该模板的文件结构如下:

    \begin{table}
        \renewcommand{\arraystretch}{1.5}
        \begin{tabular}{ll}
            \toprule
            文件路径 & 说明 \\
            \midrule
            \lstinline|main.tex| & 主文件,包含了整个文档的结构和内容 \\
            \lstinline|init.tex| & 初始化文件,用于导入\lstinline|init/|文件夹中用于初始化的内容 \\
            \lstinline|reference.bib| & \lstinline|.bib|文件,用于放置bib格式的参考文献 \\
            \lstinline|init/cmds.tex| & 用于定义可能用到的命令 \\
            \lstinline|init/code.tex| & 用于定义不同风格的代码格式 \\
            \lstinline|init/color.tex| & 用于定义可能用到的颜色 \\
            \lstinline|init/format.tex| & 用于定义文档若干元素的格式,包括标题、副标题等 \\
            \lstinline|init/pkg.tex| & 用于导入所需要的宏包 \\
            \bottomrule
        \end{tabular}
    \end{table}

\end{frame}

\section{准备工作}

\makesection

\subsection{模板使用准备工作}

\begin{frame}{\insertsection}{\insertsubsection}
    \begin{block}{模板使用准备工作}
        如果您需要使用东南大学主题,可以跳过该部分,直接使用改模板。如果您需要使用其他学校或单位的主题,可以参考本节的内容进行修改。
        \begin{enumerate}
            \item 首先将您的学校和单位的logo放置在路径\lstinline|figs/logo.jpg|下。
            \item 然后,修改\lstinline|init/color.tex|中的\lstinline|themecolor|为您的学校或单位的主题色,并将\lstinline|themered|修改为与主题色相搭配的强调色。
            \item 这样,模板就设置好啦!
        \end{enumerate}
    \end{block}
\end{frame}

\section{模板使用}

\makesection

\subsection{小节和目录}

\begin{frame}[fragile]{\insertsection}{\insertsubsection}

    \begin{block}{小节的使用}
        \begin{enumerate}
            \item 使用\lstinline|\section{}|和\lstinline|\subsection{}|命令可以生成小节和子小节。
            \item 在小节后使用\lstinline|\makesection[width]|命令可以生成小节的封面,其中\lstinline|width|表示小节封面中横线的长度,默认值为0.4。
            \begin{lstlisting}[style=latex]
\section{直接使用}
\subsection{小节和目录}
\makesection
            \end{lstlisting}

        \end{enumerate}
    \end{block}

    \begin{block}{目录的使用}
        \begin{enumerate}
            \item 使用下面命令可以生成目录页。
            \begin{lstlisting}[style=latex]
\begin{frame}
    \frametitle{目录}
    \setlength{\parskip}{0.2em} % 设置段间距
    \tableofcontents
\end(*@@*){frame}
            \end{lstlisting}
        \end{enumerate}

    \end{block}

\end{frame}

\subsection{页面和块}
\begin{frame}[fragile]{\insertsection}{\insertsubsection}
    \begin{block}{页面的使用}
        \begin{enumerate}
            \item 使用下面命令可以生成一个页面,其中标题和副标题分别为小节的标题和子小节的标题。
            \begin{lstlisting}[style=latex]
\begin{frame}{\insertsection}{\insertsubsection}
    % 页面内容
\end(*@@*){frame}
            \end{lstlisting}
            \item 也可以使用\lstinline|\begin{frame}{title}{subtitle}|来自定义标题和副标题的内容,其中副标题可以省略。当标题和副标题都省略时,页面不会显示页眉。
            \item 当然,也可以使用\lstinline|\begin{frame}[plain]|来生成一个无页眉页脚的页面,一般用于封面页和结尾页。
            \item 使用\lstinline|\begin{frame}[fragile]{title}{subtitle}|,当该页面需要插入代码块时。
        \end{enumerate}
    \end{block}
\end{frame}

\begin{frame}[fragile]{\insertsection}{\insertsubsection}
    \begin{block}{块的使用}
        \begin{enumerate}
            \item 使用下面命令可以生成一个块。
            \begin{lstlisting}[style=latex]
\begin{block}{title}
    % 块内容
\end(*@@*){block}
            \end{lstlisting}
            \item 在块的前后使用\lstinline|\colorlet{themecolor}{somecolor}|命令可以修改块的主题颜色。
        \end{enumerate}
    \end{block}

    \colorlet{themecolor}{themered}
    \begin{block}{红色的块}
        \begin{enumerate}
            \item 这是一个主题颜色为红色的块。
            \item 下面是使用红色的块的代码。
            \begin{lstlisting}[style=latex]
\colorlet{themecolor}{themered}
\begin{block}{title}
    % 块内容
\end(*@@*){block}
\colorlet{themecolor}{themegreen}
            \end{lstlisting}
        \end{enumerate}
    \end{block}
    \colorlet{themecolor}{themegreen}

\end{frame}

\subsection{列表}

\begin{frame}{\insertsection}{\insertsubsection}

    \begin{block}{列表的使用}
        \begin{enumerate}
            \item 使用\lstinline|itemize|环境可以生成无序列表。
            \item 使用\lstinline|enumerate|环境可以生成有序列表。
        \end{enumerate}
    \end{block}

    \begin{block}{列表示例}

        \begin{itemize}
            \item 这是一个无序列表。
            \item 这是一个无序列表。
            \begin{itemize}
                \item 这是一个无序列表的子列表。
                \begin{itemize}
                    \item 这是一个无序列表的子列表的子列表。
                    \item 这是一个无序列表的子列表的子列表。
                \end{itemize}
                \item 这是一个无序列表的子列表。
            \end{itemize}
        \end{itemize}

        \begin{enumerate}
            \item 这是一个有序列表。
            \item 这是一个有序列表。
            \begin{enumerate}
                \item 这是一个有序列表的子列表。
                \begin{enumerate}
                    \item 这是一个有序列表的子列表的子列表。
                    \item 这是一个有序列表的子列表的子列表。
                \end{enumerate}
                \item 这是一个有序列表的子列表。
            \end{enumerate}
        \end{enumerate}


    \end{block}

\end{frame}

\subsection{强调和注释}

\begin{frame}[fragile]{\insertsection}{\insertsubsection}

    下面是一些强调和注释的使用方法\footnote{使用\lstinline|\footnote{}|可以插入脚注}。
    \begin{table}
        \renewcommand{\arraystretch}{1.5}
        \setlength{\tabcolsep}{3mm}
        \begin{tabular}{llll}
            \toprule
            命令 & 效果 & 命令 & 效果\\
            \midrule
            \lstinline|\textsf{}| & \textsf{无衬线} & \lstinline|\textbf{}| & \textbf{加粗} \\
            \lstinline|\textrm{}| & \textrm{衬线} & \lstinline|\texttt{}| & \texttt{等宽} \\
            \lstinline|\uline{}| & \uline{下划线} & \lstinline|\uwave{}| & \uwave{波浪线} \\
            \lstinline|\sout{}| & \sout{删除线} & \lstinline|\emph{}| & \emph{强调} \\
            \lstinline|\highlight{}| & \highlight{高亮} & \lstinline|\ulhighlight{}| & \ulhighlight{下划线高亮} \\
            \lstinline|\stronghighlight{}| & \highlight{高亮} & \lstinline|\ulhighlight{}| & \ulstronghighlight{下划线强调高亮} \\
            \bottomrule
        \end{tabular}
    \end{table}
    \begin{myquote}
        这是一个引用,但是我习惯用它来表示注释。
    \end{myquote}

    \begin{callout}{Callout}
        这里可以用来解释演示文稿中出现的名词,或与主题关系不大的其他延伸内容。为了显得这个块好看,所以我这里需要多打一些字。
    \end{callout}

\end{frame}

\subsection{图片和表格}

\begin{frame}[fragile]{\insertsection}{\insertsubsection}

    \begin{block}{图片的使用}
        \begin{enumerate}
            \item 插入图片的命令如下。
            \begin{lstlisting}[style=latex]
\begin{figure}
    \centering
    \includegraphics[width=0.6\textwidth]{example.jpg}
    \caption{图片示例}
    \label{fig:example}
\end(*@@*){figure}
            \end{lstlisting}
            \begin{figure}
                \centering
                \includegraphics[width=0.5\textwidth]{figs/example.jpg}
                \caption{图片示例}
                \label{fig:example}
            \end{figure}
        \end{enumerate}
    \end{block}

\end{frame}

\begin{frame}[fragile]{\insertsection}{\insertsubsection}

    \begin{block}{表格的使用}
        \begin{enumerate}
            \item 插入表格的命令如下。
        \begin{lstlisting}[style=latex]
\begin{table}
    \centering
    \caption{表格示例}\label{tab:example}
    \begin{tabular}{cc}
        \toprule
        % 表头
        \midrule
        % 表格内容
        \bottomrule
    \end{tabular}
\end(*@@*){table}
        \end{lstlisting}

        \begin{table}
            \centering
            \caption{表格示例}\label{tab:example}
            \begin{tabular}{cccc}
                \toprule
                第一列 & 第二列 & 第三列 & 第四列 \\
                \midrule
                1 & 2 & 3 & 4 \\
                5 & 6 & 7 & 8 \\
                \bottomrule
            \end{tabular}
        \end{table}
        \end{enumerate}
    \end{block}
\end{frame}

\subsection{公式和代码块}

\begin{frame}[fragile]{\insertsection}{\insertsubsection}
    \label{frame:eq_and_code}

    \begin{block}{公式的使用}
        插入公式的效果如下。
        \begin{equation}
            \label{eq:example}
            i\hbar\frac{\partial \psi}{\partial t}
= \frac{-\hbar^2}{2m} \left(
\frac{\partial^2}{\partial x^2}
+ \frac{\partial^2}{\partial y^2}
+ \frac{\partial^2}{\partial z^2}
\right) \psi + V \psi.
        \end{equation}
    \end{block}

    \begin{block}{代码块的使用}
        \begin{enumerate}
            \item 使用\lstinline|\lstlisting{}|命令可以插入行内代码。
            \item 插入行间代码的命令和效果如下。
            \begin{lstlisting}[style=latex]
\begin{lstlisting}[style=latex]
% 代码块的内容
\end(*@@*){lstlisting}
            \end{lstlisting}
        \end{enumerate}
    \end{block}
\end{frame}

\subsection{参考文献和引用}

\begin{frame}[fragile]{\insertsection}{\insertsubsection}

    \begin{block}{参考文献的使用}
        \begin{enumerate}
            \item 首先将参考文献的信息保存在\lstinline|reference.bib|文件中。
            \item 使用\lstinline|\footfullcite{}|命令可以插入脚注引用。
            \item 例如,Kopka等人写了一本关于\LaTeX{}的书\footfullcite{kopka2003guide}。
        \end{enumerate}
    \end{block}

    \begin{block}{引用的使用}
        \begin{enumerate}
            \item 使用\lstinline|\ref{}|命令可以引用图片、表格和公式。
            \item 使用\lstinline|\pageref{}|命令可以引用页面。
            \item 例如,我可以引用第\pageref{frame:eq_and_code}页的公式(\ref{eq:example})。
        \end{enumerate}
    \end{block}

\end{frame}

\section{总结}

\subsection{总结}

\makesection

\begin{frame}{\insertsection}{\insertsubsection}
    \begin{block}{总结}
        \begin{enumerate}
            \item 我们提供了一个简单的\LaTeX{}模板。
            \item 该模板包含了常用的功能,如插入图片、表格、公式和代码块等。
            \item 没了!
        \end{enumerate}
    \end{block}

    \begin{block}{致谢}
        \begin{enumerate}
            \item 感谢大家的支持!
            \item 本模板参考了elegant sildes\footnote{https://www.overleaf.com/latex/templates/elegant-slides/yfqyhpprvdmg}模板。
        \end{enumerate}
    \end{block}
\end{frame}

\begin{frame}{参考文献}
    \printbibliography
\end{frame}

\end{document}